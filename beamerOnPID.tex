\documentclass[10pt]{ctexbeamer}

\usetheme{Madrid}
\usecolortheme{default}

\usepackage{amsmath}
\usepackage{hyperref}

\title{Analog PID Controller Design with Op-Amps}
\subtitle{From control intuition to circuit blocks}
\author{Zhicheng Sun}
\institute{Analog Circuits Course Project}
\date{\today}
\begin{document}
\begin{frame}
	\titlepage
\end{frame}

\begin{frame}{Motivation (Why analog PID?) }
	\begin{itemize}
		\item I usually implement PID in software; this project tries to \textbf{build it in analog circuitry}.
		\item Goal: a simple structure that is \textbf{fast, low-cost, and requires no programming}.
		\item Most importantly, it helps me understand what \textbf{integrators and differentiators} mean in real circuits.
	\end{itemize}
\end{frame}

\begin{frame}{PID in one equation}
	\small
	\begin{block}{Error definition}
		\vspace{2pt}
		\begin{equation*}
			e(t) = r(t) - y(t)
		\end{equation*}
	\end{block}
	\begin{block}{Control law}
		\vspace{-2pt}
		\begin{equation*}
			u(t) = K_p e(t) + K_i \int_{0}^{t} e(\tau)\, d\tau + K_d \frac{de(t)}{dt}
		\end{equation*}
	\end{block}
	\vspace{2pt}
	Intuition: do not only look at the current error; combine \textbf{present}, \textbf{accumulated}, and \textbf{trend} information.
\end{frame}

\begin{frame}{What P / I / D actually do}
	\begin{itemize}
		\item \textbf{P (Proportional):} stronger error \(\Rightarrow\) stronger correction.
			\begin{itemize}
				\item Too small: slow response and noticeable steady-state error.
				\item Too large: overshoot or even oscillation.
			\end{itemize}
		\item \textbf{I (Integral):} accumulates error to remove steady-state error.
			\begin{itemize}
				\item Risk: integral windup when the actuator saturates.
			\end{itemize}
		\item \textbf{D (Derivative):} uses the error trend to reduce overshoot.
			\begin{itemize}
				\item Risk: sensitive to high-frequency noise \(\Rightarrow\) use band-limited differentiation.
			\end{itemize}
	\end{itemize}
\end{frame}

\begin{frame}{Analog implementation: block view}
	\begin{enumerate}
		\item \textbf{Error detector:} a differential amplifier computes \(e(t)=r(t)-y(t)\).
		\item \textbf{Three parallel paths:}
			\begin{itemize}
				\item P path: inverting amplifier (gain set by resistor ratio)
				\item I path: op-amp integrator (RC time constant)
				\item D path: band-limited differentiator (practical noise control)
			\end{itemize}
		\item \textbf{Summing \& shaping:} inverting summer + output limiting to reduce windup.
	\end{enumerate}
\end{frame}

\begin{frame}{Mapping parameters to components}
	\small
	\begin{block}{Proportional path}
		\vspace{-2pt}
		\begin{equation*}
			G_P(s) = -\frac{R_f}{R_{in}} \quad \Rightarrow \quad K_p \approx \frac{R_f}{R_{in}}
		\end{equation*}
	\end{block}
	\begin{block}{Integral path}
		\vspace{-2pt}
		\begin{equation*}
			G_I(s) = -\frac{1}{RCs} \quad \Rightarrow \quad K_i \approx \frac{1}{RC}
		\end{equation*}
	\end{block}
	\begin{block}{Practical notes}
		\begin{itemize}
			\item Add a large resistor in parallel with the integrator capacitor to reduce DC drift.
			\item Add limit/clamp at the output to mitigate integral windup.
			\item Use a band-limited differentiator to avoid noise amplification.
		\end{itemize}
	\end{block}
\end{frame}

\begin{frame}{Simulation plan \& conclusion}
	\begin{block}{What to check in simulation}
		\begin{itemize}
			\item Step response: overshoot, settling time, and steady-state error.
			\item Internal signals: outputs of P / I / D branches and the summed output.
			\item Saturation behavior: does integral windup appear and how do clamps help?
		\end{itemize}
	\end{block}
	\begin{block}{Takeaway}
		Analog PID is feasible and intuitive: gains map directly to resistor ratios and RC time constants.
		The key engineering issues are \textbf{noise} (especially D) and \textbf{saturation} (especially I).
	\end{block}
\end{frame}
\end{document}
