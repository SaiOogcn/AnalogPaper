\documentclass[a4paper,12pt]{article}
\usepackage[utf8]{inputenc}
\usepackage[T1]{fontenc}
\usepackage{amsmath}
\usepackage{graphicx}
\usepackage{geometry}
\usepackage{circuitikz} % 用于绘制电路图
\usepackage{float}
\usepackage{hyperref}

\geometry{left=2.5cm,right=2.5cm,top=2.5cm,bottom=2.5cm}

\title{基于运算放大器的模拟PID控制器设计与分析}
\author{你的名字 \\ 模拟电路课程设计}
\date{\today}

\begin{document}

\maketitle


\section{引言 (Introduction)}
简述控制系统的重要性,PID控制在工业中的广泛应用,以及使用模拟电路实现PID控制器的意义(如:响应速度快、成本低、无需编程等)。

\section{PID控制理论基础 (Theoretical Background)}
\subsection{PID控制原理}
PID控制器的输出 $u(t)$ 与输入误差 $e(t)$ 的关系为:
\begin{equation}
    u(t) = K_p e(t) + K_i \int_{0}^{t} e(\tau) d\tau + K_d \frac{de(t)}{dt}
\end{equation}
其中,$K_p$ 为比例系数,$K_i$ 为积分系数,$K_d$ 为微分系数。

\section{模拟电路设计 (Circuit Design)}
本设计主要由四个模块组成:误差检测、比例环节、积分环节、微分环节以及加法器环节。

\subsection{比例环节 (Proportional Stage)}
使用反相放大器实现。增益 $A_p = -\frac{R_f}{R_{in}}$。
% 可以在这里插入 circuitikz 代码画图

\subsection{积分环节 (Integral Stage)}
使用米勒积分器电路实现。输出电压 $v_o = -\frac{1}{RC} \int v_{in} dt$。

\subsection{微分环节 (Derivative Stage)}
使用微分器电路实现。输出电压 $v_o = -RC \frac{dv_{in}}{dt}$。
\textit{注意:实际设计中通常需串联小电阻以抑制高频噪声。}

\subsection{加法与反相环节 (Summing and Inverting Stage)}
将P、I、D三路的输出通过反相加法器汇总。由于前级均为反相,此处加法器可再次反相以恢复正极性(取决于系统需求)。

\section{参数计算与仿真 (Simulation and Results)}
\subsection{参数选择}
列出选用的电阻、电容值,以及对应的 $K_p, K_i, K_d$ 计算值。

\subsection{仿真波形}
展示输入阶跃信号时,P、I、D各环节的输出波形,以及总输出波形。

\section{结论 (Conclusion)}
总结设计的优缺点,以及未来改进方向。

\begin{thebibliography}{99}
\bibitem{ref1} 模拟电子技术基础教材.
\bibitem{ref2} 相关电路设计手册.
\end{thebibliography}

\end{document}
