\documentclass[a4paper,12pt]{ctexart}
%\usepackage[utf8]{inputenc} % ctexart 已经处理了编码,通常不需要这两个包
%\usepackage[T1]{fontenc}
\usepackage{amsmath}
\usepackage{graphicx}
\usepackage{geometry}
\usepackage{circuitikz} % 用于绘制电路图
\usepackage{float}
\usepackage{hyperref}

\geometry{left=2.5cm,right=2.5cm,top=2.5cm,bottom=2.5cm}

\title{基于模拟电路的PID控制器设计与分析}
\author{孙智诚 \\ 模拟电路课程设计}
\date{\today}

\begin{document}

\maketitle

\begin{abstract}
过往的学习和实践中,笔者往往通过算法来实现PID控制器,本文意欲以模拟电路的形式,构建一种响应更精确、更迅速,成本更低的PID控制器,因其结构简单、不用编程,
故可以在一些较为基础的场景进行应用,同时也能用以让笔者更加深刻地了解放大电路中积分器、微分器的实际意义。
\end{abstract}



\section{PID控制理论}

PID控制器的输出 $u(t)$ 与输入误差 $e(t)$ 的关系为:
\begin{equation}
    u(t) = K_p e(t) + K_i \int_{0}^{t} e(\tau) d\tau + K_d \frac{de(t)}{dt}
\end{equation}
其中,$K_p$ 为比例系数,$K_i$ 为积分系数,$K_d$ 为微分系数。也就是说,控制器拿到误差 $e(t)$ 后,并不是只关注当前的误差状态,而是把误差拆成三种不同角度的量来处理,然后把三部分结果叠加成最终的控制输出 $u(t)$。

\subsection{误差与闭环的基本逻辑}
在闭环控制中,误差一般定义为
\begin{equation}
    e(t) = r(t) - y(t)
\end{equation}
其中 $r(t)$ 为期望输入,$y(t)$ 为实际反馈值。PID的目的并不复杂:通过合理设计 $u(t)$ 的大小与方向,让系统输出 $y(t)$ 尽可能贴近 $r(t)$。

\subsection{比例项 \texorpdfstring{\(P\)}{P}}
比例项为 $u_P(t) = K_p e(t)$。它的意义很直观:误差越大,纠正力度越大。
因此 $K_p$ 往往是调参时首先关注的变量。
当 $K_p$ 较小时,系统响应偏慢,输出跟随设定值不够积极,可能出现较明显的稳态误差;
当 $K_p$ 逐渐增大时,响应会更快,但过大时容易带来超调甚至振荡。

在实际电路中,比例项对应的就是一个线性放大环节;如果后续环节本身存在饱和或限幅,那么单纯增大 $K_p$ 往往并不能无限提高效果,反而可能把系统推向不稳定。

\subsection{积分项 \texorpdfstring{\(I\)}{I}}
积分项为
$
    u_I(t) = K_i \int_{0}^{t} e(\tau) d\tau
$,
它的核心作用是消除稳态误差,只要误差长期不为零,积分就会不断累加,直到把误差推回到接近零的范围。
但积分也有明显的副作用:当系统存在输出限幅、执行器能力不足或起始误差过大时,积分项可能在短时间内累积过多,造成较大的超调;这种现象在控制中常被称为积分windup或者超调。
因此,在模拟电路实现时通常会配合一定的限幅措施,例如在积分电容并联大电阻,或在输出端加钳位,以避免积分项无限累积。

\subsection{微分项 \texorpdfstring{\(D\)}{D}}
微分项为
$u_D(t) = K_d \frac{de(t)}{dt}$,
它反映的是误差变化的快慢。当误差变化很快时,微分项会给出一个减缓其变化速率的纠正量,从而改善系统的动态过程,例如降低超调、加快收敛。
然而,微分对高频噪声非常敏感:噪声往往表现为快速抖动,经过微分后会被放大,导致控制输出抖动甚至引起电路自激。
因此工程上常用“带限微分”,即在微分网络中加入电阻或电容,形成高频滚降,而不是理想微分。

\subsection{本设计采用PID的动机}
综合来看,$P$ 负责提供基本控制力度,$I$ 用于消除稳态误差,$D$ 则在动态过程中起到抑制超调与改善响应的作用。本文后续将围绕模拟电路实现的可行性,分别给出比例、积分、微分环节的电路结构与参数设计方法,并通过仿真验证其在典型输入(如阶跃)下的响应特性。


\section{模拟电路设计}
本节将基于前文的控制理论基础,在本学期模拟电路课程学习的知识的基础上寻找对应的
解决方案,设计符合PID控制器基本原理的模拟电路,并在此基础上尝试进行优化,以满足
实际的使用需求。
\subsection{误差检测与输入调理}
误差 $e(t)$ 通常由设定值与反馈值之差得到:$e(t)=r(t)-y(t)$。在电路上可用运放差动放大器实现该减法运算。
为保证后级线性工作,本设计在误差输入端预留了限幅与简单滤波(例如输入串联电阻 + 小电容到地),用于抑制高频毛刺对微分通道的影响。

\subsection{比例环节 \texorpdfstring{\(P\)}{P} (Proportional Stage)}
比例环节用运放反相放大器实现,其传递函数可写为:
\begin{equation}
    G_P(s) = -\frac{R_f}{R_{in}}
\end{equation}
因此比例系数的等效关系为 $K_p = \frac{R_f}{R_{in}}$(符号取决于系统定义与求和方式)。
在实际搭建时,常用电位器作为 $R_f$ 或 $R_{in}$ 的一部分,以便于在实验中对 $K_p$ 做连续调节。

\subsection{积分环节 \texorpdfstring{\(I\)}{I} (Integral Stage)}
积分环节采用经典运放积分器:输入端为电阻 $R$,反馈为电容 $C$。理想情况下:
\begin{equation}
    G_I(s) = -\frac{1}{RCs}
\end{equation}
对应到PI/PID中的参数关系为 $K_i = \frac{1}{RC}$。
需要指出的是,理想积分器对直流增益无穷大,实际电路中会因运放失调、偏置电流等导致输出缓慢漂移并最终饱和。因此工程上通常在反馈电容两端并联一个较大的电阻 $R_b$,形成“带泄放的积分器”,使低频增益有限,从而提升长期稳定性。

\subsection{微分环节 \texorpdfstring{\(D\)}{D} (Derivative Stage)}
微分环节的理想形式为 $G_D(s)=K_d s$,但理想微分会显著放大高频噪声。为避免输出抖动与自激,本设计采用带限微分(超前环节)的实现方式:输入串联 $C_d$ 与 $R_d$ 形成高通特性,反馈端再通过小电容/电阻形成高频滚降,使得微分只在目标频段内发挥作用。
在参数设计上,微分通道更关注有效频带与噪声抑制,而不是盲目追求“微分越强越好”。

\subsection{求和与输出整形 (Summing and Output Shaping)}
将 $P$、$I$、$D$ 三路输出用运放反相加法器进行求和,得到控制输出 $u(t)$。反相加法器的一个优点是各路权重可以直接用输入电阻比值体现,便于参数标定与调整。
此外,考虑到实际执行机构(电机驱动、功率级等)通常存在输入范围限制,输出端可加入对称限幅(如二极管钳位或运放供电限幅下的饱和保护),以减少因过大控制量导致的积分饱和现象。

\section{参数计算与仿真 (Simulation and Results)}
\subsection{参数选择}
列出选用的电阻、电容值,以及对应的 $K_p, K_i, K_d$ 计算值。

\subsection{仿真波形}
展示输入阶跃信号时,P、I、D各环节的输出波形,以及总输出波形。

\section{结论 (Conclusion)}
总结设计的优缺点,以及未来改进方向。

\begin{thebibliography}{99}
\bibitem{ref1} 模拟电子技术基础教材.
\bibitem{ref2} 相关电路设计手册.
\end{thebibliography}

\end{document}
