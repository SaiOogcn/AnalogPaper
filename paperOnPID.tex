\documentclass[a4paper,12pt]{ctexart}
%\usepackage[utf8]{inputenc} % ctexart 已经处理了编码,通常不需要这两个包
%\usepackage[T1]{fontenc}
\usepackage{amsmath}
\usepackage{graphicx}
\usepackage{geometry}
\usepackage{circuitikz} % 用于绘制电路图
\usepackage{float}
\usepackage{hyperref}

\geometry{left=2.5cm,right=2.5cm,top=2.5cm,bottom=2.5cm}

\title{基于分立晶体管(BJT)的模拟PID控制器设计与分析}
\author{孙智诚 \\ 模拟电路课程设计}
\date{\today}

\begin{document}

\maketitle

\begin{abstract}
本文旨在设计一种基于分立双极性结型晶体管(BJT)的模拟PID控制器。不同于常见的运算放大器方案,本设计从底层出发,利用差分放大电路实现误差检测,利用共射极放大电路及其变体构建比例、积分和微分环节。文章详细分析了各级电路的静态工作点(Q点)设置与动态参数计算,并通过仿真验证了该分立元件电路在控制系统中的有效性。该设计有助于深入理解模拟电路中晶体管的放大与运算功能。
\end{abstract}


\section{引言 (Introduction)}
简述控制系统的重要性,PID控制在工业中的广泛应用,以及使用模拟电路实现PID控制器的意义(如:响应速度快、成本低、无需编程等)。

\section{PID控制理论基础 (Theoretical Background)}
\subsection{PID控制原理}
PID控制器的输出 $u(t)$ 与输入误差 $e(t)$ 的关系为:
\begin{equation}
    u(t) = K_p e(t) + K_i \int_{0}^{t} e(\tau) d\tau + K_d \frac{de(t)}{dt}
\end{equation}
其中,$K_p$ 为比例系数,$K_i$ 为积分系数,$K_d$ 为微分系数。

\section{模拟电路设计 (Circuit Design)}
本设计完全采用分立BJT元件构建。主要包含误差检测(差分放大)、PID运算单元(共射/共基电路)及信号合成(加法器)模块。

\subsection{误差检测级 (Error Detector)}
采用长尾对(Long-tailed Pair)差分放大电路。
\begin{itemize}
    \item \textbf{电路结构}:两个匹配的NPN管(如2N3904),发射极共用一个恒流源(或大电阻)接地。
    \item \textbf{功能}:输入端分别接设定值($V_{set}$)和反馈值($V_{fb}$),输出差模电压 $V_{err} \propto (V_{set} - V_{fb})$。
\end{itemize}

\subsection{比例环节 (Proportional Stage)}
采用带发射极负反馈电阻的共射放大电路。
\begin{itemize}
    \item \textbf{原理}:电压增益 $A_v \approx -\frac{R_C}{R_E}$。
    \item \textbf{调节}:通过改变 $R_E$ 或 $R_C$ 来调节比例系数 $K_p$。
\end{itemize}

\subsection{积分环节 (Integral Stage)}
采用改进型的米勒积分器(Miller Integrator)。
\begin{itemize}
    \item \textbf{电路结构}:在共射放大电路的集电极与基极之间跨接电容 $C$。
    \item \textbf{原理}:利用米勒效应,将电容等效到输入端,形成积分特性。
\end{itemize}

\subsection{微分环节 (Derivative Stage)}
采用RC高通网络配合射极跟随器。
\begin{itemize}
    \item \textbf{电路结构}:输入信号先经过 $C-R$ 微分网络,再进入高输入阻抗的射极跟随器(Common Collector)进行缓冲,防止负载效应。
\end{itemize}

\subsection{加法环节 (Summing Stage)}
采用电阻加法网络接入共射放大器基极。
\begin{itemize}
    \item \textbf{原理}:将P、I、D三路的输出通过电阻汇聚到加法级晶体管的基极,利用基极电流的叠加实现信号相加。
\end{itemize}

\section{参数计算与仿真 (Simulation and Results)}
\subsection{参数选择}
列出选用的电阻、电容值,以及对应的 $K_p, K_i, K_d$ 计算值。

\subsection{仿真波形}
展示输入阶跃信号时,P、I、D各环节的输出波形,以及总输出波形。

\section{结论 (Conclusion)}
总结设计的优缺点,以及未来改进方向。

\begin{thebibliography}{99}
\bibitem{ref1} 模拟电子技术基础教材.
\bibitem{ref2} 相关电路设计手册.
\end{thebibliography}

\end{document}
